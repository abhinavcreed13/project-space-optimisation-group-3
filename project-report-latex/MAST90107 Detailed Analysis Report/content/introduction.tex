The university has spent its second-largest expense on space allocation and the arrangement of meeting rooms and toilets has long been recognized as a major concern in campus planning. It is important to ensure optimal space utilization as under-utilization of these facilities entails extra cost penalties for maintenance. In this project, the space arrangement of staff meeting rooms and student toilets will be optimized by proposing solutions that can efficiently use the current supply of resources. Generally, the number of existing meeting rooms and toilets is considered as “supply” while the number of staff and enrolled students are considered as “demand”. By analyzing space, employee, and timetabling data, we will first explore if the supply meets current demand, then propose different predictive models that will help our client to suggest the usage of current resources more efficiently.

Our client for this project is the Spatial Analytics and Space Management department of the University of Melbourne. This department works in future space design, better space allocation, and optimizing usage of resources for the university. As stated by our client, the expected outcome of this project can be summarized as:
\vspace{-2mm}
\begin{itemize}
    \item We need to suggest how the \textbf{space arrangement} of meeting rooms and toilets can be optimized, and advise how the overall space on campus can be better planned.
    \vspace{-2mm}
    \item We are expected to deliver a \textbf{detailed analysis report}. The analysis is expected to be conducted by campus, by building, and by different meeting rooms and toilet types, etc. The report should include interactive maps, charts with the interpretation of findings.
    \vspace{-2mm}
    \item We need to use different \textbf{analytical methods} such as spatial analysis, correlation analysis, etc.
    \vspace{-10mm}
    \item We need to provide \textbf{reasonable recommendations} of space optimization opportunities based on the analysis.
\end{itemize}
\vspace{-2mm}

From the data science perspective, this project involves extensive exploratory data analysis of supply and demand, complicated data mutations, joins, and preprocessing. We also need to perform a correlation analysis among different factors and spatial analysis using QGIS. To suggest the current usage of resources more effectively, we need to construct predictive statistical models using well-defined constraints. The models pose an integration challenge of python models with QGIS spatial layers. Also, models are supposed to be extremely generic so that they can provide support for analyzing any building on any campus. Moreover, identifying appropriate factors for correlation analysis from the provided data is a difficult and daunting task that we'll be exploring in this project.

This report is organized as follows: Section 2 describes problem objective and formulation. We present our results for exploratory data analysis and factor analysis in section 3. Section 4 discusses proposed methodologies and section 5 shows findings for spatial and floor algorithms. Finally, an analysis of the algorithm is presented in section 6 and we conclude our work in section 7.