In this section, we will be reviewing literature and research articles that follow closely to our provided problem and proposed solution. In 1987, Golden, Bruce, and others introduced the term \texttt{The Orienteering Problem} which is analogous to an outdoor sport played in heavily forested areas. According to them, forests are having "control points" associated with a score and the task is to visit each control point (node $n$) from a starting point with the objective of maximizing the cumulative collected score\cite{IPP2}. They proposed the complexity of solving this problem to be NP-hard and provided a gravity heuristic for relaxing and solving this problem\cite{IPP2}. In 2014, Yu, Jingjin, and others proposed a novel non-linear extension to the orienteering problem (OP) which they termed as \texttt{Correlated Orienteering Problem (COP)}\cite{IPP3}. They assumed spatial correlations among the reward providing nodes and proposed quadratic extension for the OP to incorporate such correlations in the informative path planning phase\cite{IPP3}.

Due to the NP-hard nature of this problem, S. Arora and others proposed a randomized algorithm for informative path planning with budget constraints in 2017\cite{IPPArora}. Their research inspired the problem formulation and proposed algorithm introduced in this report. They transformed the OP domain into a constraint satisfaction problem and proposed several versions of the randomized anytime algorithm to provide the most rewarding path respecting the budget constraint\cite{IPPArora}. In addition to this, Wei, Yongyong, and others introduced a Reinforcement Learning based approach for solving an informative path planning problem in 2020\cite{IPP4}. Their work inspired the cost-reward representation idea that we have implemented for formulating our problem. Finally, we explored the mathematical understanding of Min-Max Heaps and Generalized Priority Queues as introduced by Atkinson and others\cite{heap1}. This helped us to grasp the space and time complexity domain that motivated the design of our proposed algorithm.
