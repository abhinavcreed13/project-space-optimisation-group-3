\thispagestyle{empty}
\topskip0pt
\vspace*{\fill}
\begin{abstract}
Optimally utilizing the space for meeting rooms and toilet facilities is extremely important for effective campus planning and proper space allocation. In this project, we were tasked with the objective of suggesting reasonable recommendations of space optimization opportunities by performing analysis of the provided data using data science tools and techniques. In order to provide these recommendations and give a deeper analysis of the data, we transformed these general objectives into a prediction problem that lies in the domain of Informative Path Planning (IPP).

In this report, we perform the proper transformation of the objectives and represent the problem formally. We are going to transform the provided campus space into several specialized graphs with respect to different buildings and floors. We will propose a novel non-randomized anytime orienteering algorithm for finding k-optimal goals that maximize reward on a specialized graph with budget constraints. We will also explain the cost and reward function modeling with their corresponding hyperparameters tuning process. Using this algorithm, we will present the results of several buildings with a supply-demand problem. This report also provides an analysis of our proposed algorithm. Our experimental results suggest reasonable space optimization opportunities across different campuses of the University of Melbourne.
\end{abstract}
\vspace*{\fill}
\pagebreak