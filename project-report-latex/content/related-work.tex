In this section, we have researched multiple literature resources which tackle similar problems. The optimization of meeting rooms and toilets allocation is a classical scheduling problem and considered to be NP-complete\cite{np_problem}. Since students’ activities are limited by their selected subjects, staff are usually based in a single location, and other than the physical distance, the actual utilization of meeting rooms and toilets should also be considered. Our goal is taking all of those variables into consideration and see how the arrangement and maintenance services can be optimised given current supply and demand.

Ahmed Wasfy and Fadi A. Aloul proposed a complete approach using integer linear programming (ILP) to solve similar scheduling problem\cite{ilp}. The goal of this algorithm is satisfying all of the university's constraints to assigning courses to classrooms and optimizing the utilization of existing facilities effectively and efficiently.  

Burke et al.\cite{burke2010decomposition} also proposed an integer linear programming method to solve this problem. This method decomposes the problem into multiple sub-problems. In each sub-problem, only one part of the optimization constraint or criteria is considered. These sub-problems will be used to work outbounds in their respective optimization criterion. Finally, the ultimate solution to the problem is incorporated by each solution of the sub-problems.

Vermuyten et al.\cite{vermuyten2016developing} proposed a two-stage approach to optimize demand allocation using integer linear programming. Firstly, the approach focuses on event assignments to rooms and time slots, and the second focuses on the reassignments of meeting rooms to events with previous time slots from the previous stage. 

Beyrouthy et al.\cite{doi:10.1057/palgrave.jors.2602523} researched when planning to build a campus, how teaching space can be utilized to improve usage efficiency. The study shows that most rooms have overcapacity, which means the supply (seat number) is greater than the demand (students’ number). 

Song et al.\cite{song2018iterated} proposed three stages iterative algorithm, the three stages are initialization, intensification, and diversification. This algorithm is used to solve the course scheduling problem. The initialization stage is to use a greedy algorithm to find a roughly feasible solution to schedule the maximum possible events given specific constraints and variables. The second stage is to use a simulated annealing method to find a local optimal solution. The diversification stage implements random perturbations (exchange some rooms) to improve the result. The final result developed by this algorithm is used as the new initial scenario and implement these stages iteratively.

Greedy heuristics have been implemented to course timetabling and meeting room scheduling problems successfully. For example, “Greedy Randomized Adaptive Search Procedure (GRASP)” \cite{moura2010grasp}. The greedy algorithm is designed through a cost function that takes all constraints violated into accounts. However, this algorithm cannot guarantee to reach an optimal solution. Furthermore, this algorithm is not considering room optimization as its constrains.
