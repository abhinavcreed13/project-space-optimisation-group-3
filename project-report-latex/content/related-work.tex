Multiple literature resources tackle the similar problems have been researched. The optimization of meeting rooms and toilets allocation is a classical scheduling problem and considered to be NP-complete \cite{np_problem} because the search space for possible solutions grows exponentially with the number of events. Since students’ activities are limited by their selected subjects, staff are usually based in a single location, and other than the physical distance, the actual utilization of meeting rooms and toilets should also be considered. Our goal is taking all of those variables into consideration and see how the arrangement and maintenance services can be optimised given current supply and demand.

Several algorithms have been proposed to solve these problems. Most of these algorithms are based on local search techniques, which is meaning searching for local optimal solution, such as simulated annealing, and tabu search. Satisfiability or guarantee cannot be proved by such algorithms, so it cannot confirm that a solution is optimal. In other words, by implementing those algorithms, the solution found cannot be guaranteed to be the best possible optimization.

Ahmed Wasfy and Fadi A. Aloul \cite{ilp} proposed a complete approach using integer linear programming (ILP) to solve similar scheduling problem. The goal of this algorithms is while satisfying all of the university constraints to assign courses to classrooms and optimizing the utilization of existing facilities effectively and efficiently. This method shows how to formulate the university schedule as an integer linear programming problem. The ILP models are solved using advanced generic-based and Boolean satisfiability based ILP solvers. 

Burke et al.\cite{burke2010decomposition} also proposed an integer linear programming method to solve timetabling problems. This method decomposes the problem into multiple sub-problems. In each sub-problem, only one part of the optimization constraint or criteria is considered. These sub-problems will be used to work out bounds in their respective optimization criterion. Finally, the ultimate solution to the problem is incorporated by each solution of the sub-problems.

Vermuyten et al.\cite{vermuyten2016developing} proposed a two-stage approach to optimize demand allocation using integer linear programming. Firstly, the approach focuses on event assignments to rooms and time slots, and the second focuses on reassignments of meeting rooms to events with previous time slots from the previous stage. The primary criterion for optimization is the second stage, which assigns meeting rooms in order to avoid congestion. From student perspective, some constrains are also added to the integer linear programming implementation in regard to the compaction of the scheduling. The integer linear programming tries to avoid schedules with relative long time without any events for the students, hence, maximizing the usage of resources. This type of variable is important for students who lives far from University particularly. However, the proposed method has failed to reduce the overall number of long gaps between events. For example, some students only have one event in some days, or two events are far in different venues. This is a common practice in decomposition since it reduces the overall complexity of main problem. 

Beyrouthy et al.\cite{doi:10.1057/palgrave.jors.2602523} researched when planning building a campus, how teaching space can be utilized in order to improve usage efficiency. The study shows that most rooms have overcapacity, which means the supply (seats number) is greater than the demand (students’ number). Apart from this, there are strong evidences to indicate that the location of the room has a significant impact on room utilization since both students and staff prefer certain locations.

Song et al.\cite{song2018iterated} proposed a three stages iterative algorithm, the three stages are initialization, intensification, and diversification. This algorithm is to solve the course scheduling problem. The initialization stage is to use a greedy algorithm to find a roughly feasible solution to schedule the maximum possible events given specific constrains and variables. The second stage is to use a simulated annealing method to find a local optimal solution. The diversification stage implements random perturbations (exchange some rooms) to improve the result. The final result developed by this algorithm is used as the new initial scenario and implement these stages iteratively.

Greedy heuristics have been implemented to course timetabling and meeting room scheduling problems successfully. For example, “Greedy Randomized Adaptive Search Procedure (GRASP)” \cite{moura2010grasp}. The greedy algorithm is designed through a cost function that taking all constraints violated into accounts. However, this algorithm cannot guarantee to reach an optimal solution. Furthermore, this algorithm is not considering room optimization as its constrains.
