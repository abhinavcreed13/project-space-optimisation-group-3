The university has spent its second-largest expense to space allocation and the arrangement of meeting rooms and toilets has long been recognised as a major concern in campus planning. It is important to ensure optimal space utilisation as under-utilisation of these facilities entails extra cost penalties for maintenance. In this project, the space arrangement of staff meeting rooms and student toilets will be optimised by proposing solutions that can be efficiently use the current supply of resources. Generally, the number of existing meeting rooms and toilets is considered as “supply” while the number of staffs and enrolled students are considered as “demand”. By analysing space, employee and timetabling data, we will firstly explore if the supply meets current demand, then propose different predictive models that will help our client to suggest the usage of current resources more efficiently.

Our client for this project is the Spatial Analytics and Space Management department of the University of Melbourne. This department works in future space design, better space allocation and optimising usage of resources for the university. As stated by our client, the expected outcome from this project can be summarized as:
\vspace{-2mm}
\begin{itemize}
    \item We need to suggest how the \textbf{space arrangement} of meeting rooms and toilets can be optimised, and advise how the overall space on campus can be better planned.
    \vspace{-2mm}
    \item We are expected to deliver a \textbf{detailed analysis report}. The analysis is expected to be conducted by campus, by building, and by different meeting room and toilet types, etc. The report should include interactive maps, charts with interpretation of findings.
    \vspace{-2mm}
    \item We need to use different \textbf{analytical methods} such as spatial analysis, correlation analysis, etc.
    \vspace{-10mm}
    \item We need to provide \textbf{reasonable recommendations} of space optimisation opportunities based on the analysis.
\end{itemize}
\vspace{-2mm}
From the data science perspective, this project involves extensive exploratory data analysis of supply and demand, complicated data mutations, joins and preprocessing. We also need to perform correlation analysis of different factors and spatial analysis using QGIS. In order to suggest current usage of resources more effectively, we need to do predictive statistical modelling using well-defined constraints. This kind of modelling poses an integration challenge of python models with QGIS spatial layers. We also need to make our models extremely generic so that they can provide support for analyzing any building in any campus. Moreover, identifying appropriate factors for correlation analysis from the provided data is a difficult and daunting task that we'll be exploring in this project.

This report is organised as follows: Section 2 describes relevant work. We present the results of our basic analysis using the provided datasets, including data preprocessing, correlations and spatial analysis in section 3. Section 4 discusses our solutions for suggesting how to use the current supply of resources more efficiently. Finally, a summarized timeline of part-1 and a clear plan for next semester is discussed in section 5.