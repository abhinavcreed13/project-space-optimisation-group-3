The university has spent its second-largest expense to space allocation and the arrangement of meeting rooms and toilets has long been recognised as a major concern in campus planning. It is important to ensure optimal space utilisation as under-utilisation of these facilities entails extra cost penalties for maintenance; while shortages of these facilities may bring inconveniences for both staffs and students. In this project, the space arrangement of staff meeting rooms and student toilets will be optimised based on a supply and demand analysis. Generally, the number of existing meeting rooms and toilets is considered as “supply” while the number of staffs and enrolled students is considered as “demand”. By analysing space, employee and timetabling data, we will firstly explore if the supply meets current demand, then provide reasonable suggestions for optimisation. 

From a data science perspective, this project contains a large amount of space data, which involves spatial analysis technique and software that we have never learnt before and it becomes the main difficulty for us. Also, students and staffs have different activity patterns in the campus - students’ activities largely depend on their chosen subjects, however, staffs usually stay in fixed locations. The analysis is supposed to provide appropriate spaces that meet various needs for users. Furthermore, it is necessary to consider flexibility and adaptability to ongoing changes when planning since students’ class registrations and staffs’ arrangements might vary from semester to semester.

This report is organised as follows: Section 2 describes relevant work. We present the result for basic analysis of given datasets, including data preprocessing and visualisation in section 3. Section 4 discusses our methodologies for approaching the problem. And a clear plan for next semester is summarised in section 5.